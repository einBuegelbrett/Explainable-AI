\documentclass[ngerman,14pt,aspectratio=1610]{beamer}

% Imports
\usepackage[utf8]{inputenc}
\usepackage[T1]{fontenc}
\usepackage{babel}
\usepackage{graphicx}
\usepackage{multicol}
\usepackage{textpos} % Logo in frametitle
\usepackage{ifthen} % Für das \ifthenelse
\usepackage{totcount} % Für Counter, die in main.aux gespeichert werden
\usepackage{forest}

% Startpfad für Bilder setzen
\graphicspath{ {./images/} }

% DHBW Logos
\newcommand{\dhbwlangtrans}{\includegraphics[width=5cm]{dhbw_lang_trans}}
\newcommand{\dhbwkurzweiss}{\includegraphics[width=0.5\paperwidth]{dhbw}}

% Subsections zählen
\def\secname{Gliederung} % Der Titel der jeweiligen Section wird in secname gespeichert, damit innerhalb der Section der counter mit dem name erhöht werden kann
\let\oldsection\section
\renewcommand{\section}[1]{
	\oldsection{#1}
	\newtotcounter{#1}
	\def\secname{#1}
}
\let\oldsubsection\subsection
\renewcommand{\subsection}[1]{
	\oldsubsection{#1} 
	\stepcounter{\secname}
}

% Nummern bei Frametitle
\newcommand{\secpagnum}{\insertsectionnumber.\thesubsection}
\newcommand{\lastpagenum}{1}
\let\oldframetitle\frametitle
\renewcommand{\frametitle}[1]{
	\ifthenelse{\equal{\lastpagenum}{\insertslidenumber}}{
		\subsection{}
		\renewcommand{\lastpagenum}{\insertslidenumber}
	}{
		% Hier soll nichts passieren, damit bei gleichen slides der title nicht neu gesetzt wird
	}
	
	\ifthenelse{\totvalue{\secname}>1}{
	\oldframetitle{\secpagnum~#1}
	}{
	\oldframetitle{\insertsectionnumber~#1}
	}
}

% Multicol column balancing ausschalten
\raggedcolumns

% Basis-Theme
\usetheme[progressbar=frametitle, block=fill, sectionpage=none]{metropolis} % Hier kann die progressbar deaktiviert/replaziert werden
\setbeamertemplate{frame numbering}[counter]
\useoutertheme{metropolis}
\useinnertheme{metropolis}
\usefonttheme{metropolis}
\setbeamercolor{background canvas}{bg=white}

% Sections im ToC Nummerieren
\setbeamertemplate{section in toc}[sections numbered]
\setbeamertemplate{subsection in toc}[subsections numbered]

% Zeilenabstand ToC
\makeatletter
\patchcmd{\beamer@sectionintoc}
{\vfill}
{\vskip\itemsep}
{}
{}
\makeatother 

% DHBW Farben
\definecolor{dhbw_grau}{RGB}{93,104,110}
\definecolor{dhbw_rot}{RGB}{227,6,19}
\definecolor{dunkelgrau}{RGB}{41,55,67}
\definecolor{hellgrau}{RGB}{239,241,242}

% DHBW Farben einbauen
\setbeamercolor{progress bar}{fg=dhbw_rot, bg=hellgrau}
\setbeamercolor{normal text}{fg=dunkelgrau} %bg hier macht block bg
\setbeamercolor{alerted text}{fg=dhbw_rot}
\setbeamercolor{frametitle}{fg=dhbw_rot,bg=hellgrau}
\setbeamercolor{title}{fg=dhbw_rot}
\setbeamercolor{subtitle}{fg=dunkelgrau}
\setbeamercolor{section title}{fg=dhbw_rot}
\setbeamercolor{institute}{fg=dhbw_rot}

% Logo in frametitle
\addtobeamertemplate{frametitle}{}{%
	\begin{textblock*}{5cm}(\textwidth-4cm,-1.3cm)
		\dhbwlangtrans
	\end{textblock*}}

% Datum in der Fußzeile
\setbeamertemplate{frame footer}{
	\hskip 2.5em \insertdate
} 

% Endseite
\newcommand{\finalpage}[2][Vielen Dank für ihre Aufmerksamkeit!]{
\metroset{sectionpage=none}
\setbeamertemplate{frame numbering}[none]
\setbeamertemplate{frame footer}{}
\oldsection*{#1}
\begin{frame} \vspace{60pt}
	\sectionpage
	\vspace{10pt}
	\centering
	#2
\end{frame}
}

% Daten für die Titelseite
\title{Explainable-AI - Post-Hoc-Analyse - Erkennung von Hirntumoren}
%\subtitle{subtitle}
\author{Sven Sendke}
\institute[DHBW Stuttgart]{Duale Hochschule Baden-Württemberg Stuttgart}
\date{09.12.2024}	
\titlegraphic { 
	\begin{tikzpicture}[overlay,remember picture]
		\node[right=0.5cm] at (current page.155){
			\dhbwlangtrans
		};
	\end{tikzpicture}
}

\regtotcounter{section}
\regtotcounter{subsection}

\begin{document}
	
	\begin{frame}[plain,noframenumbering]
		\titlepage
	\end{frame}

	\section{Gliederung}
	
	\begin{frame}[t]{Gliederung} \vspace{20pt}
		\begin{columns}[T, onlytextwidth]
			\column{0.5\textwidth}
			\linespread{1.5}
			\tableofcontents
			\column{0.5\textwidth}
		\end{columns}
	\end{frame}
	
	\metroset{sectionpage=simple} % Erst hier definiert, damit die Gliederung keine Sectionpage hat
	\section{Einleitung}
		\begin{frame}[t]{Einleitung - Gehirntumor} \vspace{10pt}
			\begin{columns}[T, onlytextwidth]
				\column{0.45\textwidth}
				\begin{itemize}
					\item \textbf{Abnormales} Wachstum von \textbf{Zellen} im Gehirn
					\item Deutschland \textbf{jährlich} etwa \textbf{8.000 Menschen} neu daran (Robert Koch Instituts Berlin)
					\item \textbf{Zeitaufwändige} und \textbf{teure} Untersuchungsmethode
				\end{itemize}
				
				\column{0.45\textwidth}
				\includegraphics[width=\linewidth]{brain\_tumor\_1}
			\end{columns}
		\end{frame}
		
		\begin{frame}[t]{Einleitung - Herausforderungen} \vspace*{\fill}
				\begin{itemize}
					\item Einsatz von CNNs in der  \textbf{medizinischen Bildsegmentierung}
					\item Erkennung von Hirntumoren in MRT-Bildern ist schwierig aufgrund der \textbf{komplexen Gehirnstruktur}.
					\item Schwierigkeit, die Gründe für die Antwort von CNN zu verstehen
				\end{itemize}
		\end{frame}
		
		%Challenges: 
		%Detecting brain tumors from MRI images is challenging due to the complex structure of the brain. [1] In this context, convolutional neural networks (CNNs) have proven effective in a wide range of computer vision tasks. Recent advances in semantic segmentation have further enabled their application to medical image segmentation. [2] The aim of this project is to understand the specific strengths and difficulties that artificial intelligence, and CNNs in particular, face in brain tumor detection.
		
		\begin{frame}[t]{Einleitung - Forschungsfrage}
			\vspace*{\fill}
			Erstellung eines CNNs zur \textbf{Erkennung von Hirntumoren} und Anwendung von \textbf{Layer-wise Relevance Propagation} (LRP) zur Analyse der Relevanz der einzelnen Pixel für die \textbf{Entscheidungsfindung} des Netzwerks, um herauszufinden, welche \textbf{Muster} bei der Erkennung geholfen haben.
		\end{frame}
		
		\begin{frame}[t]{Einleitung - Datensatz}
			\begin{columns}[T, onlytextwidth]
				\column{0.45\textwidth}
				\vspace{50pt}
					\includegraphics[width=\linewidth]{kaggle\_1}
					\includegraphics[width=\linewidth]{kaggle\_2}
					
				\column{0.45\textwidth}
				\vspace{70pt}
					https://www.kaggle.com/
					datasets/preetviradiya/brian
					-tumor-dataset/data
			\end{columns}
		\end{frame}
		
		\begin{frame}[t]{Einleitung - Datensatz} 
			\vspace*{\fill}
			\begin{columns}[T, onlytextwidth]
				\column{0.45\textwidth}
				Probleme mit dem Datensatz:
				\begin{itemize}
					\item Doppelte Bilder
					\item Einige Bilder sind kleine Variationen
				\end{itemize}
				
				\column{0.45\textwidth}
				\begin{block}{Achtung!}
					Auf der Seite steht, es seien Röntgenbilder, aber es sind MRT / CT Bilder!!!!
				\end{block}
			\end{columns}
		\end{frame}
	
	\section{Projekt}
		\begin{frame}[t]{Projekt - GitHub Repository} \vspace{20pt}
			\includegraphics[width=\linewidth]{github}
		\end{frame}
		
		\begin{frame}[t]{Projekt - CNN} \vspace{10pt}
			\centering
			\includegraphics[width=\linewidth, height=0.75\textheight, keepaspectratio]{net}
		\end{frame}
		
		\begin{frame}[t]{Projekt - Konfusionsmatrix} \vspace{10pt}
			\centering
			\includegraphics[width=\linewidth, height=0.75\textheight, keepaspectratio]{confusion\_matrix}
		\end{frame}
		
	\section{LRP}
		\begin{frame}[t]{LRP - Theorie} \vspace{20pt}
			Layer-wise Relevance Propagation (LRP):
			\begin{itemize}
				\item Relevanz von jeden Neuron in ein Neuronales Netz
				\item Foward pass für eine Prediction
				\item  output is backward propagated layer by layer (Spezielle LRP decomposition rules)
				\item Heatmap mit den contribution of individual input features (e.g., pixels) to the prediction
			\end{itemize}
		\end{frame}
		
		\begin{frame}[t]{LRP - Theorie - Bild} \vspace*{\fill}
			Hier noch einmal als Bild:
			\includegraphics[width=\linewidth]{lrp\_procedure}
		\end{frame}
		
		\begin{frame}[t]{LRP - Dateistruktur} \vspace{20pt}		
			\begin{figure}[h]
				\centering
				\scalebox{0.9}{
					\begin{forest}
						for tree={
							font=\ttfamily,
							grow'=0,
							child anchor=west,
							parent anchor=south,
							anchor=west,
							calign=first,
							edge path={
								\noexpand\path [draw, \forestoption{edge}]
								(!u.south west) +(7.5pt,0) |- node[fill,inner sep=1.25pt] {} (.child anchor)\forestoption{edge label};
							},
							before typesetting nodes={
								if n=1
								{insert before={[,phantom]}}
								{}
							},
							fit=band,
							before computing xy={l=15pt},
						},
						s=0.1
						[...
						[lrp
						[\_\_init\_\_.py]
						[lrp.py]
						[lrp\_filter.py]
						[lrp\_layers.py]
						]
						[...]
						]
					\end{forest}
				}
			\end{figure} 
		\end{frame}
		
		\begin{frame}[t]{LRP - Implementierung}
			 \vspace*{\fill}
			 \centering
			\includegraphics[width=\linewidth, height=0.75\textheight, keepaspectratio]{lrp\_code}
		\end{frame}
		
	\section{Beobachtungen}
		\begin{frame}[t]{Beobachtungen - Gehirntumor}			
			\begin{columns}[T, onlytextwidth]
				\column{0.20\textwidth}
				\vspace{60pt}
					Auge/Tumor? \\
					$\rightarrow$ wenige Datensätze mit Augen
				
				\column{0.80\textwidth}
				\vspace{20pt}
					\includegraphics[width=\linewidth]{eyes\_brain\_tumor}
			\end{columns}
		\end{frame}
		
		\begin{frame}[t]{Beobachtungen - Gehirntumor}
			\begin{columns}[T, onlytextwidth]
				\column{0.20\textwidth}
				\vspace{90pt}
				Gehirntumor erkannt
				
				\column{0.80\textwidth}
					\vspace{20pt}
					\includegraphics[width=\linewidth]{brain\_tumor\_2}
			\end{columns}
		\end{frame}
		
		\begin{frame}[t]{Beobachtungen - Gesund} 
			\begin{columns}[T, onlytextwidth]
				\column{0.20\textwidth}
				\vspace{50pt}
					Bild mit wenigen Details\\
					$\rightarrow$  Hülle ohne Auffälligkeiten 
				
				\column{0.80\textwidth}
					\vspace{20pt}
					\includegraphics[width=\linewidth]{reconize\_h\_1}
			\end{columns}
		\end{frame}
		
		\begin{frame}[t]{Beobachtungen - Gesund} 
			\begin{columns}[T, onlytextwidth]
				\column{0.20\textwidth}
					\vspace{60pt}
					Bild mit vielen Details\\
					$\rightarrow$ Gehirnform entscheidend
				\column{0.80\textwidth}
					\vspace{20pt}
					\includegraphics[width=\linewidth]{reconize\_h\_2}
			\end{columns}
		\end{frame}
		
	\section{Fazit}
		\begin{frame}[t]{Fazit - LRP Vorteile/Nachteile} \vspace{20pt}
			\begin{columns}[T, onlytextwidth]
				\column{0.45\textwidth}
				Vorteile:
				\begin{itemize}
					\item Relevante Merkmale als Heatmap
					\item Die Gesamtrelevanz auf allen Ebenen bleibt erhalten
				\end{itemize}
				
				\column{0.45\textwidth}
				Nachteile
				\begin{itemize}
					\item Ähnliche Merkmale können unabhängig von der untersuchten Klasse hervorgehoben werden
					\item Knoten mit irrelevanten Informationen können hervorgehoben werden
				\end{itemize}
			\end{columns}
		\end{frame}
		
		\begin{frame}[t]{Fazit - Reflexion} \vspace{20pt}
			\begin{columns}[T, onlytextwidth]
				\column{0.45\textwidth}
				Gut:
				\begin{itemize}
					\item LRP erzeugt klarere und gezieltere Heatmaps, die relevante Merkmale in den Eingabedaten hervorheben.
					\item test2
					\item test3
				\end{itemize}
				
				\column{0.45\textwidth}
				Schlecht:
				\begin{itemize}
					\item Datensatz
					\item Viel zu Optimieren: zuerst 70\% dann 95\%
				\end{itemize}
			\end{columns}
		\end{frame}
		
		\begin{frame}[t]{Fazit - Literaturverzeichnis} \vspace{20pt}
			\includegraphics[width=\linewidth]{references}
		\end{frame}
	
	\finalpage{\inserttitlegraphic}
\end{document}